\documentclass[12pt]{article}
\title{Elektrotechnik 2 \\ Notizen }
\author{K L U K}
\usepackage{amsmath}
\usepackage{MnSymbol}
\begin{document}
\maketitle 
\newpage
\section{19. Globale und lokale Eigenschaften magnetischer Felder}
\subsection{Der Satz vom magnetischen Hüllenfluss}
Globale Eigenschaft magn. Flussverteilung:
\begin{equation}\label{key}
\phi(\partial \mathcal{V}) = 0
\end{equation}
Darstellung magn. Flüsse als Flächensumme der Flussdichte: 
\begin{equation}\label{key}
\phi(\mathcal{A}) = \int_{\mathcal{A}} B_n dA
\end{equation}
als Kurvensumme des Vektorpotentials: 
\begin{equation}\label{key}
\phi(\mathcal{A})=\int_{\partial \mathcal{A}} A_s ds
\end{equation}
Kommentar von Peter: 
\begin{equation}\label{key}
d\vec{s} = \vec{e_s} ds  \quad \vec{e_s}\cdot \vec{A} = |\vec{A}|\, cos(\alpha) = A_s 
\end{equation}
\paragraph{Verhalten der magnetischen Flussdichte an einer Sprungfläche}

\begin{equation}\label{key}
G^+-G^-=\lsem G \rsem
\end{equation}
\paragraph{Verhalten des magnetischen Vektorpotentials an einer Sprungfläche} %19:21
An einer Sprungfläche ist die \textbf{Tangenzialkomponente } des magnetischen Vektorpotential stetig. 
\newpage
% % % % % % % % % % % % % % % % % % % %






% % % % % % % % % % % % % % % % % % % %
\section*{28. Energie im Elektromagnetismus}
\subsection{Energiespeicherung }
Der ideale Kondensator speichert die Energie mit folgender Beziehung: 
\begin{equation}\label{key}
I = C\dot{U},\quad P(t) = U(t) I(t) = C U(t) \dot{U}(t)
\end{equation}
\noindent
Der Energieerhalt wird beschrieben durch: 
\begin{equation}\label{key}
W_c = \frac{CU^2}{2}=\frac{QU}{2}=\frac{Q^2}{2C}
\end{equation}

\paragraph{Eigenschwingungen}
Eigenschwingung bedeutet ein ständiger Energieaustausch zwischen unabhängigen Energiespeichern. 

\begin{align}
\dot{U}=-I/C, \quad  U(0) = \hat{U} \\
\dot{I}=U/L, \quad I(0) = 0 
\end{align}
\end{document}